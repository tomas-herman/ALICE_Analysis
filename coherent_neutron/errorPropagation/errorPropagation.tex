\documentclass[a4paper,12pt]{article}

\usepackage{graphicx}
\usepackage{grffile}
\usepackage{boxhandler}
\usepackage[utf8]{inputenc}
%\usepackage[czech,english]{babel}
\usepackage{mathptmx} %ALICE style J/psi
\DeclareMathAlphabet{\mathcal}{OMS}{cmsy}{m}{n}
\usepackage{amsmath} % matematické výrazy
\usepackage{amssymb} % matematické výrazy
%\usepackage[Symbol]{upgreek} % recke verze matematickych reckych symbolu
\usepackage{array}
\usepackage[dvipsnames]{xcolor}   % dovolí pužívat barvicky
\usepackage{wrapfig}   % text obiha kolem obrazku
\usepackage{caption}   % caption zarovnavani
\usepackage{fancyhdr}  % záhlaví zápatí
\usepackage{gensymb}   % stupne Celsia
\usepackage{subcaption} %for sub figures
\usepackage{incgraph,tikz} %obrazek pres celou stranu
\usepackage{mathrsfs}  %znak pro integrovanou luminozitu
\usepackage{hyperref} %odkazy jak hypertextové tak na tabulky
\usepackage{appendix}

\setlength{\intextsep}{1cm}

\newcommand*\dif{\mathop{}\!\mathrm{d}} %new command for writing differentials
\newcommand*\Dif[1]{\mathop{}\!\mathrm{d^#1}} %new command for writing differentials of ghigher orders
\newcommand*\jpsi{\mathrm{J}/\psi} % Command for Jpsi typing
\newcommand*\syst{(\mathrm{syst})} %Command for sys
\newcommand*\stat{(\mathrm{stat})} %Commad for stat


% completely avoid orphans (first lines of a new paragraph on the bottom of a page)
\clubpenalty=10000

% completely avoid widows (last lines of paragraph on a new page)
\widowpenalty=10000


%  Formát papíru a odsazení od jeho okrajů
\usepackage[a4paper]{geometry}
%\geometry{verbose,tmargin=2.5cm,bmargin=2cm,lmargin=3.5cm,rmargin=2cm,headheight=0.8cm,headsep=1cm,footskip=0.5cm}
\geometry{bmargin=4cm, lmargin=3.5cm}

\usepackage{a4wide}  %Formát: evropský standard

\parindent=0pt % odsazení 1. řádku odstavce
\parskip=5pt   % mezera mezi odstavci
\frenchspacing % aktivuje použití některých českých typografických pravidel


\begin{document}

%%%%%%%%%%%%%%%%%%%%%%%%%%%%%%%%
% begin main text
%%%%%%%%%%%%%%%%%%%%%%%%%%%%%%%%

Source of data for the plotting and computation is Table 4 of the paper draft. 

\begin{table}[h!]
\begin{tabular}{rrrrrrr}
\hline
 $W_{\gamma Pb,n}$ (GeV) & $\sigma$ ($\mu$b) & unc. ($\mu$b)  & corr. ($\mu$b) & mig. ($\mu$b) & flux frac. ($\mu$b)  & IA ($\mu$b) \\
\hline
19.12 &8.80 &0.30 &0.67 &0.12 &0.06 &13 \\
813.05 &60.23 &19.94 &8.21 &15.22 &8.81 & 196  \\
\hline
24.55 &13.86 &0.23 &1.10 &0.19 &0.11  & 18\\
633.21 &47.80 &6.50 &10.76 &8.73 &5.16  & 167\\
\hline
31.53 &16.78 &0.59 &1.31 &0.35 &0.25  & 22\\
493.14 &46.00 &6.32 &5.30 &6.72 &4.22  & 142\\
\hline
97.11 &20.21 &5.14 &3.13 &7.53 &3.81  & 49\\
160.10 &27.03 &7.40 &4.96 &10.94 &5.44  & 68\\
\hline
124.69 &24.10 &0.70 &1.36 &0.23 &0.15 & 58\\
\hline
\end{tabular}
\end{table}

Putting these values to arrays which are used in the code.

The energy:

$W_{\gamma Pb,n}$    = \{ 19.12, 813.05, 24.55, 633.21, 31.53, 493.14, 97.11, 160.1, 124.69  \};

------------

The data cross section and the uncertainty due to the flux:

$\sigma^{Data}_{\gamma Pb}$[9]  = \{ 8.80, 60.23, 13.86, 47.80, 16.78, 46.00, 20.21, 27.03, 24.10  \};

$\Delta{\sigma^{Data}_{\gamma Pb}}$[9]  = \{ 0.06, 8.81,  0.11, 5.16, 0.25, 4.22, 3.81, 5.44, 0.15  \};

------------

The IA cross section and the corresponding uncertainty:

$\sigma^{IA}_{\gamma Pb}$[9]  = \{ 13.20, 196.34, 17.95, 166.89, 22.46, 141.85, 49.16, 68.19, 57.92  \};

$\Delta{\sigma^{IA}_{\gamma Pb}}$[9]  = \{ 0.66, 9.82, 0.90, 8.35, 1.12, 7.09, 2.46, 3.41, 2.90  \};

------------

The formula to compute the nuclear suppression factor is:

\begin{equation}
S_{Pb} = \sqrt[]{\frac{\sigma^{Data}_{\gamma Pb}}{\sigma^{IA}_{\gamma Pb}}}
\end{equation}

Then the error propagation formula is (considering the errors are uncorrelated)

\begin{equation}
\Delta S_{Pb} = \sqrt[]{ (\Delta{\sigma^{Data}_{\gamma Pb}})^2 \left(\frac{\partial S}{\partial \sigma^{Data}_{\gamma Pb}}\right)^2 + (\Delta{\sigma^{IA}_{\gamma Pb}})^2 \left(\frac{\partial S}{\partial \sigma^{IA}_{\gamma Pb}}\right)^2 }
\end{equation}

Doing the derivatives (omitting the minus signs as they will be squared anyway)

\begin{equation}
\Delta S_{Pb} = \sqrt[]{ (\Delta{\sigma^{Data}_{\gamma Pb}})^2 \left(\frac{1}{2 \sigma^{IA}_{\gamma Pb}  \sqrt[]{\frac{\sigma^{Data}_{\gamma Pb}}{\sigma^{IA}_{\gamma Pb}}} }\right)^2 + (\Delta{\sigma^{IA}_{\gamma Pb}})^2 \left(\frac{ \sigma^{Data}_{\gamma Pb} }{ 2 (\sigma^{IA}_{\gamma Pb})^2  \sqrt[]{\frac{\sigma^{Data}_{\gamma Pb}}{\sigma^{IA}_{\gamma Pb}}} }\right)^2 }
\end{equation}

Squaring and simplifying the brackets to get the final result

\begin{equation}
\Delta S_{Pb} = \sqrt[]{ (\Delta{\sigma^{Data}_{\gamma Pb}})^2 \left(\frac{1}{4 \sigma^{IA}_{\gamma Pb} \sigma^{Data}_{\gamma Pb}  }\right) + (\Delta{\sigma^{IA}_{\gamma Pb}})^2 \left(\frac{ \sigma^{Data}_{\gamma Pb} }{ 4 (\sigma^{IA}_{\gamma Pb})^3  }\right) }
\end{equation}




%%%%%%%%%%%%%%%%%%%%%%%%%%%%%%%%
% end main text 
%%%%%%%%%%%%%%%%%%%%%%%%%%%%%%%%

\end{document}